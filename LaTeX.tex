\documentclass[a4paper,12pt]{article}
\usepackage[T2A]{fontenc}
\usepackage[utf8]{inputenc}
\usepackage{longtable}
\usepackage[left=2cm, right=1.5cm, top=2cm, bottom=2cm]{geometry}
\usepackage{array}
\usepackage{wrapfig}
\usepackage[warn]{mathtext}
\usepackage[russian]{babel}
\usepackage{amssymb}
\usepackage{graphicx, float, multicol, hyperref, pgfplots, amsmath}
\usepackage{pgfplots}
\pgfplotsset{compat=1.18}
\usepackage{tikz}
\usepackage{rotating}
\usepackage[english,russian]{babel}
\usepackage{amsmath,amsfonts,amssymb,amsthm,mathtools}
\usepackage{graphicx}
\usepackage{subcaption}

\begin{document}

\begin{titlepage}
	\begin{center}
		{\large МОСКОВСКИЙ ФИЗИКО-ТЕХНИЧЕСКИЙ ИНСТИТУТ (НАЦИОНАЛЬНЫЙ ИССЛЕДОВАТЕЛЬСКИЙ УНИВЕРСИТЕТ)}
	\end{center}
	\begin{center}
		{\large Физтех-школа радиотехники и компьютерных технологий (ФРКТ)}
	\end{center}


	\vspace{4.5cm}
	{\huge
		\begin{center}
			{\bf Отчёт о взятии производной произвольной функции}\
		\end{center}
	}
	\vspace{2cm}
	\begin{flushright}
		{\LARGE Автор: \\ Ковалев Михаил Андреевич \\
			\vspace{0.2cm}
			Группа Б01-502}
	\end{flushright}
	\vspace{8cm}
	\begin{center}
		г. Долгопрудный\\
		\today
	\end{center}
\end{titlepage}
\section{Ваше исходное выражение}

\[\frac{\left(x- (-4)\right)\cdot \left(y+ 6.72\cdot z\right)}{\operatorname{sin}\left(x^{x}\right) }\]

\end{document}
